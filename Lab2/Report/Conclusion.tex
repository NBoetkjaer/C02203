\chapter{Conclusion}
\paragraph*{ }
The aim of this project has been to design and implement a FPGA based hardware accelerator, that performs a Sobel filtering on an image.
The edge detection project has been solved in a structured design process, using RTL methodologies to design block diagrams and  ASMD charts. The general strategy, throughout this project, has been to implement a relatively basic design which is gradually  optimized, to give better performance.
\paragraph*{ }
All the implementations employ a border replication technique, in order to generate an output image with the same dimension as the input image.\\
The first design, which was using a sliding window technique, was able to process $4.1$\textit{M pixel/sec.} Because this design involved four memory transaction per pixel pair this corresponds to data rate of $16.4MB/sec$, when considering both reading and writing from the memory. In asynchronous mode the external memory is capable of a data rate of $25MB/sec.$, which means that this design was not able to use the full potential of the asynchronous memory.\\
In the second design we introduced a scanline buffering mechanism using block RAM, that resulted in a throughput of $12.2M pixel/sec.$ which was a significant improvement. Because this design requires only two memory transactions (read and write) per pixel pair, it equals a data rate of $24.4MB/sec.$ and thereby exploits the full asynchronous bandwidth of the external memory. \\
Our third and last design, which was never fully implemented due to lack of time, involved the synchronous mode of the memory, running at $80MHz$. The theoretically throughput of this design is close to $80M pixel/sec.$ which is equivalent to a data rate of $160MB/sec.$ and thereby utilizing the full bandwidth of the synchronous memory mode.
\paragraph*{ }
For all of the implemented designs, it is clear that the memory access is the dominant limitation in terms of achieving a high throughput. The maximum clock rate of the synthesized implementations are $102.8Mz$ and $89.9MHz$ respectively, which is much higher the asynchronous memory mode but only slightly larger than the synchronous burst mode.


%Overall the project have showed that an edge detection accelerator can indeed be implemented on FPGA how ever with the available hardware in this project it would not be possible to use it for real time calculation on a HD video stream. Since that would require the extern ram to run at $156Mhz$ giving $25.07$ \textit{frames pr. second} for a Full HD resolution.

\newpage