\usepackage[latin1]{inputenc}% Skal passe til editorens indstillinger
\usepackage{cite}
\usepackage{amsfonts}
\usepackage{amsmath}
\usepackage{amssymb}
\usepackage{graphicx} % inds�ttelse af billeder
\usepackage[english]{babel}% eng. overskrifter
\usepackage{subfig}
\usepackage{float}
\usepackage{verbatim} % useful for program listings
\usepackage{fancyvrb}
\usepackage{lmodern} % vektor fonte
\usepackage{listings} %include c code
\usepackage[T1]{fontenc} % fonte (output)
\usepackage{fixltx2e}  % Retter forskellige bugs i LaTeX-kernen
\usepackage{endnotes} 	%puts all the 'footnotes' at the end of the document.\\
\usepackage{tikz}
\usepackage{geometry,fancyhdr}
\geometry{
 %total={210mm,297mm},
 left=22mm,
 right=20mm,
 top=30mm,
 bottom=30mm,
 }
\usepackage[final]{pdfpages}
\usepackage{lastpage}
%\usepackage[hidelinks]{hyperref}
\usepackage[hidelinks]{hyperref}
\usepackage{multirow}
\graphicspath{{billeder/}{Matlab/}} % stivej til bibliotek med figurer
\setlength\parindent{0pt}
\usepackage{tikz,pgfplots}
\usetikzlibrary{patterns}
\usepackage{circuitikz}

\usetikzlibrary{shapes,arrows}

\pgfplotsset{compat=newest} 
\pgfplotsset{plot coordinates/math parser=false}

\usepackage{tikz}
\usetikzlibrary{fit,shapes.geometric}
\usetikzlibrary{fit,shapes.misc}
\usetikzlibrary{shapes,arrows,calc,positioning}

\tikzstyle{sblock} = [rectangle, draw=black, fill=blue!20, minimum height=2em, minimum width=2em, line width=0.35mm, rounded corners]
\tikzstyle{mblock} = [rectangle, draw=black, fill=blue!20, minimum height=3em, minimum width=3em, line width=0.35mm, rounded corners]
\tikzstyle{lblock} = [rectangle, draw=black, fill=blue!20, minimum height=3em, minimum width=4em, line width=0.35mm, rounded corners]
\tikzstyle{xlblock} = [rectangle, draw=black, fill=blue!20, minimum height=5em, minimum width=8em, line width=0.35mm, rounded corners]

\tikzstyle{sum} = [circle, draw=black, fill=black!50, line width=0.3mm]
\tikzstyle{point} = [coordinate]
\tikzstyle{branch} = [circle, draw=black, fill=black, minimum size=1.5mm, inner sep=0pt]
\tikzstyle{arrow} = [->,line width=0.4mm]
\tikzstyle{line} = [-,line width=0.4mm]
\tikzstyle{pinstyle} = [pin edge={to-,thin,black}]

\newcommand{\unit}[1]{\ \mathrm{#1}}

\newcommand\marktopleft[1]{%
    \tikz[overlay,remember picture] 
        \node (marker-#1-a) at (0,1ex) {};%
}
\newcommand\markbottomright[1]{%
    \tikz[overlay,remember picture] 
        \node (marker-#1-b) at (0,0) {};%
    \tikz[overlay,remember picture,thick,dashed,inner sep=3pt]
        \node[draw,rectangle,fit=(marker-#1-a.center) (marker-#1-b.center)] {};%
 %\node[draw,ellipse,fit=(marker-#1-a.center) (marker-#1-b.center)] {};%
}

\newlength\figureheight  	
\newlength\figurewidth

\usepackage{listings}
\lstset{
  % backgroundcolor=\color{white},   % choose the background color; you must add \usepackage{color} or \usepackage{xcolor}
	basicstyle=\scriptsize,        % the size of the fonts that are used for the code
  % breakatwhitespace=false,         % sets if automatic breaks should only happen at whitespace
	breaklines=true,                 % sets automatic line breaking
  % captionpos=b,                    % sets the caption-position to bottom
	commentstyle=\color{mygreen},    % comment style
  % deletekeywords={...},            % if you want to delete keywords from the given language
  % escapeinside={\%*}{*)},          % if you want to add LaTeX within your code
  % extendedchars=true,              % lets you use non-ASCII characters; for 8-bits encodings only, does not work with UTF-8
  % frame=single,                    % adds a frame around the code
	framexleftmargin=-4pt,
  % keepspaces=true,                 % keeps spaces in text, useful for keeping indentation of code (possibly needs columns=flexible)
	keywordstyle=\color{blue},       % keyword style
  % morekeywords={*,...},            % if you want to add more keywords to the set
	numbers=left,                    % where to put the line-numbers; possible values are (none, left, right)
	numbersep=5pt,                   % how far the line-numbers are from the code
	numberstyle=\tiny\color{mygray}, % the style that is used for the line-numbers
  % rulecolor=\color{black},         % if not set, the frame-color may be changed on line-breaks within not-black text (e.g. comments (green here))
  % showspaces=false,                % show spaces everywhere adding particular underscores; it overrides 'showstringspaces'
  % showstringspaces=false,          % underline spaces within strings only
  % showtabs=false,                  % show tabs within strings adding particular underscores
  % stepnumber=2,                    % the step between two line-numbers. If it's 1, each line will be numbered
	stringstyle=\color{mymauve},     % string literal style
	tabsize=3,                       % sets default tabsize to 2 spaces
  % title=\lstname,                   % show the filename of files included with \lstinputlisting; also try caption instead of title
	xleftmargin=5pt,
	language=VHDL                 % the language of the code
}

\usepackage{framed}

\usepackage{color}
\definecolor{mygreen}{rgb}{0,0.6,0}
\definecolor{mygray}{rgb}{0.5,0.5,0.5}
\definecolor{mymauve}{rgb}{0.58,0,0.82}  